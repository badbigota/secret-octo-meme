% !TeX encoding = utf8
% !TeX program = pdflatex
% !TeXpellcheck = it_IT

\documentclass[a4paper,11pt,oneside]{article} 

\usepackage{relazioni}
\usepackage{imakeidx}
\usepackage{colortbl}
\usepackage{booktabs}
\usepackage{blindtext}
\usepackage{titletoc}
\usepackage{hyperref}
\usepackage{graphicx}
\usepackage{subcaption}
\usepackage{wrapfig}
\usepackage[export]{adjustbox}
\hypersetup{
%    colorlinks=false,
} 

\def\titolo{Esperienza guidovia}
\def\dataconsegna{20 Gennaio 2020}

\graphicspath{{Figure/}}

\begin{document}
% Franceschetto Giacomo v.2018
% !TeX encoding = utf8
% !TeX program = pdflatex
% !TeX root = ../relazione.tex

\thispagestyle{empty}
\begin{minipage}{0.51 \textwidth}
	\vspace{-1 cm}	\begin{flushleft}
		\hspace{-2cm}	\includegraphics[width=0.9 \textwidth]{logo}
	\end{flushleft}
\end{minipage}
\begin{minipage}{0.51\textwidth}
	\vspace{-1 cm}	\begin{flushright}
		\emph{Sperimentazioni 1 - AL, A.A. 2019/2020} \\
		\emph{Docente : C. Sada, Canale A-L}
	\end{flushright}
\end{minipage}
\null

\bigskip\bigskip\bigskip\bigskip\bigskip\bigskip\bigskip\bigskip\bigskip\bigskip\bigskip\bigskip\bigskip\bigskip\bigskip


\begin{center}
    {\Huge{GUIDOVIA}}
\end{center}

\begin{center}
    {\Large\emph{Relazione di laboratorio - 13 Marzo 2020}}\par
\end{center}
\par

\bigskip\bigskip\bigskip\bigskip\bigskip\bigskip\bigskip\bigskip\bigskip\bigskip\bigskip\bigskip\bigskip\bigskip\bigskip\bigskip\bigskip\bigskip\bigskip

\begin{table}[h!]
		\centering
		\begin{tabular}{rll}
			\Large\color{Pantone1807}\itshape{Tommaso Bertola}	& {1216608} &  {tommaso.bertola@studenti.unipd.it}\\[0.125 cm]
			\Large\color{Pantone1807}\itshape{Fabio Cufino}	& {1230202} &  {fabio.cufino@studenti.unipd.it}\\[0.125 cm]
			\Large\color{Pantone1807}\itshape{Marco Lorenzato}	& {1216647} &  {marco.lorenzato.2@studenti.unipd.it}\\[0.125 cm]
		\end{tabular}
	\end{table}


\tableofcontents
\addtocontents{toc}{~\hfill{Pagina}\par}
\contentsmargin{6em}
\dottedcontents{section}[1em]{\bigskip}{2em}{1pc}
\dottedcontents{subsection}[3em]{\smallskip}{3em}{1pc}
\dottedcontents{subsubsection}[5em]{\smallskip}{4em}{1pc}


\newpage

\section{Obiettivo}
%Si può dire meglio e in modo più conciso e chiaro
La verifica sperimentale del moto uniformemente accelerato di una slitta su un piano inclinato, in assenza di attrito radente, è volta alla stima dell'accelerazione  di gravità a Padova.\\



%La conseguente stima dell'attrito viscoso che l' aria imprime all'oggetto preso in considerazione e l'apporto di altre considerazioni teoriche e sperimentali comporta la correzione della stima dell’accelerazione di gravità precedentemente considerata.



%NON BISOGNA METTERE L'INTRODUZIONE TEORICA
%\section{Introduzione teorica}


\section{Apparato sperimentale}\label{section:apparato}

\begin{figure}[h!]
    \centering
    \includegraphics[width= 10cm]{DiagrammaApparato.jpg}
    \caption{Diagramma composizione apparato}
    \label{fig:apparato_sperimentale}
\end{figure}

Per lo svolgimento dell'esperienza si è fatto uso di:
\begin{enumerate}
    \item Una slitta in plexiglas rettangolare con delle sporgenze sulla parte inferiore al fine di agevolarne lo scorrimento sul piano inclinato. La slitta è eventualmente equipaggiabile con dischi di ottone per aumentarne la massa.
    \item Una guida in alluminio a sezione rettangolare e a inclinazione variabile (tramite una vite) sulla quale far scorrere la slitta. La guida è inoltre forata strategicamente su tutta la sua lunghezza per permettere l'uscita dell'aria erogata da un soffiatore a pressione regolabile. Ciò premette la conseguente formazione di un cuscino d'aria sotto la slitta durante la sua corsa al fine di eliminare il contributo dell'attrito radente.
    \item Un cronometro digitale avente sensibilità di \num{1e4}\si{s^-1}
    \item Traguardi a sensori a infrarossi posizionati perpendicolarmente alla guida per delimitare una qualsiasi porzione di spazio sulla stessa e utilizzati per azionare e fermare il cronometro al passaggio della slitta. In particolare il primo permetteva l'inizio del conteggio, fermato poi dal passaggio della slitta al secondo.
    \item Elettrocalamita posta all'inizio della guida, utilizzata nella prima parte dell'esperienza per bloccare la slitta per poi rilasciarla, nella seconda parte  per imprimere una forza  alla slitta stessa. L'elettrocalamita è azionabile manualmente da un operatore.
    \item Spessori di alluminio utilizzati per ridurre la spinta impressa dall'elettrocalamita, interponibili tra quest'ultima e la slitta.
\end{enumerate}
    

\section{Metodo}
\subsection{Orizzontalità della guida}
La fase preliminare dell'esperienza prevede di determinare l'orizzontalità della guida tramite misurazioni dirette. La procedura impiegata consiste nel verificare se la slitta rimane stabile in 3 punti differenti, a 40 cm, 80 cm e 110 cm dall'origine della guida.\\
In primo luogo si è proceduto con l'accensione del soffiatore, impostato per tutta l'esperienza alla stessa potenza, per poi posizionare la slitta nel primo punto, a 40cm. Si è poi regolata la vite affinché la slitta rimanesse stabile al fine di poter fissare un sistema di riferimento con il quale testare l'orizzontalità sugli altri due punti. Lasciando invariato il sistema di riferimento sono stati annotati i valori sulla vite per i quali, nel secondo  e nel terzo punto, la slitta rimane stabile.\\
Ipotizzando l'imprecisione delle misure precedentemente effettuate si è ripetuto il procedimento e si è verificato che per il secondo e per il terzo punto il risultato rimaneva invariato, invece per il primo punto si è misurato un nuovo valore. Data l'ipotesi, si è calcolata la media e la deviazione standard su queste ultime 3 misure effettuate.\\
La seguente tabella riporta i dati ottenuti:

\begin{table}[h]
    \centering
        \caption{Valori per l'orizzontalità della guida \\ Unità di misura: quarti di giro di vite}
\begin{tabular}{ccc|cc}
    \toprule
    $\alpha_{0}$&$\alpha_{2}$&$\alpha_{3}$&$\overline{\alpha}$&$\sigma_{\overline{\alpha}}$\\
    \midrule
0.15&0.15&0.4&0.23&0.1443375672 \\
    \bottomrule
    \end{tabular}
    \label{tab:guida_orizzontale}
\end{table}

Dati i valori ottenuti, si è assunto come zero del nuovo sistema di riferimento, considerando anche la relativa deviazione standard, la media delle misure $\overline{\alpha} \pm \sigma_{\overline{\alpha}}$, settando la ghiera.\\


\subsection{Presa dati prima esperienza}
Fissato il sistema di riferimento si è impostata la guida ad un'inclinazione di $15'$ ruotando 3 volte la vite, in quanto ciascun giro corrisponde ad una variazione  di $5'$.\\

Si sono posizionati i traguardi, inizialmente a 40cm e 50cm e, successivamente, lasciando invariata la posizione del primo, si è spostato solo il secondo a step di 10cm per volta fino ad arrivare a 110cm. \\
Per ciascun intervallo di spazio ([40cm,50cm], [40cm,60cm], \dots [40cm,110cm]) si sono prese 5 misure ripetute di intervalli di tempo, ognuna delle quali rappresentava il tempo impiegato dalla slitta per percorrere il relativo spazio.\\  %Sono stati letti i dati dal cronometro automatico, in seguito riportati in file di testo.\\
Il cronometro è stato impostato per entrambe le esperienze
Si è ripetuta la medesima operazione di presa dati variando l'inclinazione della slitta a 30' e successivamente a 45'.
Infine, operando sempre sulla slitta inclinata di 45', si è caricata la slitta con un disco di ottone al fine di incrementarne la massa.

\subsection{Presa dati seconda esperienza}
Si è portata la guida all'orizzontalità e l'elettrocalamita è stata configurata per imprimere una spinta alla slitta scarica della massa di ottone. Ciò ha consentito alla slitta di procedere per la guidovia, rallentando per le forze d'attrito viscoso dell'aria, oggetto di analisi in questa seconda esperienza.\\
I traguardi sono stati posizionati rispettivamente a 40cm e 60cm, per poi modificarne la distanza dall'origine aumentandola di 10 cm per volta arrivando fino all'intervallo 90-110cm e per ogni intervallo di spazio sono state prese 5 misurazioni temporali.\\
Si è poi ripetuto il procedimento e la conseguente presa dati, una volta con la slitta carica e una seconda volta con la slitta scarica utilizzando uno spessore di alluminio interposto tra la slitta e l'elettrocalamita.

\section{Analisi dati}
\subsection{Prima esperienza}
Si sono dapprima analizzate le misurazioni relative alla guida inclinata di 15' seguendo il procedimento riportato in seguito.\\
Per ciascun intervallo di spazio ([40cm, 50cm], [40cm, 60cm], \dots) si è calcolata la miglior stima di tendenza centrale, la media $\overline{x}$, il suo relativo errore $\sigma_{\overline{x}}$ e la deviazione standard $\sigma$. Inoltre si sono calcolate le semisomme dei tempi medi tra coppie di intervalli consecutivi, ai quali d'ora in avanti si farà riferimento con il termine \textit{tempi intermedi}, e gli errori ad essi associati tramite il teorema delle varianze \ref{}.
I dati ottenuti sono riportati in tabella \ref{tab:15_primi}


\begin{table}[]
\begin{tabular}{ll|lll|lll|lll|lll}
\toprule
\multicolumn{2}{l|}{}      & \multicolumn{3}{c|}{\textbf{15'}}                      & \multicolumn{3}{c|}{\textbf{30'}}                      & \multicolumn{3}{c}{\textbf{45'}}                       & \multicolumn{3}{c|}{\textbf{45' con Massa}}            \\ \midrule
Inizio      & Fine        & $\overline{t}$ & $\sigma$     & $\sigma_{\overline{t}}$ & $\overline{t}$ & $\sigma$     & $\sigma_{\overline{t}}$ & $\overline{t}$ & $\sigma$     & $\sigma_{\overline{t}}$ & $\overline{t}$ & $\sigma$     & $\sigma_{\overline{t}}$ \\
$[\si{cm}]$ & $[\si{cm}]$ & $[\si{sec}]$   & $[\si{sec}]$ & $[\si{sec}]$           & $[\si{sec}]$   & $[\si{sec}]$ & $[\si{sec}]$           & $[\si{sec}]$   & $[\si{sec}]$ & $[\si{sec}]$           & $[\si{sec}]$   & $[\si{sec}]$ & $[\si{sec}]$           \\ \midrule
40          & 50          & 0.6421200000   & 0.0033611000 & 0.0015031300           & 0.4528400000   & 0.0008876940 & 0.0003969890           & 0.3697000000   & 0.0004898980 & 0.0002190890           & 0.3729200000   & 0.0002167950 & 0.0000969536           \\
40          & 60          & 1.1917800000   & 0.0029149600 & 0.0013036100           & 0.8434600000   & 0.0017826900 & 0.0007972450           & 0.6892200000   & 0.0010183300 & 0.0004554120           & 0.6915000000   & 0.0006480740 & 0.0002898280           \\
40          & 70          & 1.6925200000   & 0.0052432800 & 0.0023448700           & 1.1920200000   & 0.0014923100 & 0.0006673830           & 0.9755600000   & 0.0014170400 & 0.0006337190           & 0.9782600000   & 0.0003974920 & 0.0001777640           \\
40          & 80          & 2.1431600000   & 0.0093377200 & 0.0041759500           & 1.5132400000   & 0.0029022400 & 0.0012979200           & 1.2364200000   & 0.0008843080 & 0.0003954740           & 1.2389600000   & 0.0013145300 & 0.0005878780           \\
40          & 90          & 2.5579200000   & 0.0084974700 & 0.0038001800           & 1.8019400000   & 0.0015789200 & 0.0007061160           & 1.4777600000   & 0.0016164800 & 0.0007229110           & 1.4787000000   & 0.0010000000 & 0.0004472140           \\
40          & 100         & 2.9446800000   & 0.0034607800 & 0.0015477100           & 2.0811600000   & 0.0020562100 & 0.0009195650           & 1.7026400000   & 0.0009316650 & 0.0004166530           & 1.7025400000   & 0.0003646920 & 0.0001630950           \\
40          & 110         & 3.3156400000   & 0.0065036900 & 0.0029085400           & 2.3383000000   & 0.0026172500 & 0.0011704700           & 1.9139400000   & 0.0027125600 & 0.0012131           & 1.9118800   & 0.0013065 & 0.0005842 \\
\bottomrule
\end{tabular}
\end{table}

\begin{table}[h]
\centering
\begin{tabular}{cc|ccc}
\toprule
Inizio&Fine&$\overline{t}$&$\sigma$&$\sigma_{\overline{t}}$\\
$[\si{cm}]$&[cm]&[sec]&[sec]&[sec]\\
\midrule
40 & 50  & 0.64212 & 0.0033611  & 0.00150313 \\
40 & 60  & 1.19178 & 0.00291496 & 0.00130361 \\
40 & 70  & 1.69252 & 0.00524328 & 0.00234487 \\
40 & 80  & 2.14316 & 0.00933772 & 0.00417595 \\
40 & 90  & 2.55792 & 0.00849747 & 0.00380018 \\
40 & 100 & 2.94468 & 0.00346078 & 0.00154771 \\
40 & 110 & 3.31564 & 0.00650369 & 0.00290854\\
\bottomrule
\end{tabular}
    \caption{Medie tempi a 15'}
    \label{tab:15_primi}
\end{table}

\begin{table}[h]
\centering
\begin{tabular}{cc|ccc}
\toprule
Inizio&Fine&$\overline{t}$&$\sigma$&$\sigma_{\overline{t}}$\\
$[\si{cm}]$&[cm]&[sec]&[sec]&[sec]\\
\midrule
40 & 50  & 0.45284 & 0.000887694 & 0.000396989 \\
40 & 60  & 0.84346 & 0.00178269  & 0.000797245 \\
40 & 70  & 1.19202 & 0.00149231  & 0.000667383 \\
40 & 80  & 1.51324 & 0.00290224  & 0.00129792  \\
40 & 90  & 1.80194 & 0.00157892  & 0.000706116 \\
40 & 100 & 2.08116 & 0.00205621  & 0.000919565 \\
40 & 110 & 2.3383  & 0.00261725  & 0.00117047 \\
\bottomrule
\end{tabular}
    \caption{Medie tempi a 30'}
    \label{tab:30_primi}
\end{table}

\begin{table}[h]
\centering
\begin{tabular}{cc|ccc}
\toprule
Inizio&Fine&$\overline{t}$&$\sigma$&$\sigma_{\overline{t}}$\\
$[\si{cm}]$&[cm]&[sec]&[sec]&[sec]\\
\midrule
40 & 50  & 0.3697  & 0.000489898 & 0.000219089 \\
40 & 60  & 0.68922 & 0.00101833  & 0.000455412 \\
40 & 70  & 0.97556 & 0.00141704  & 0.000633719 \\
40 & 80  & 1.23642 & 0.000884308 & 0.000395474 \\
40 & 90  & 1.47776 & 0.00161648  & 0.000722911 \\
40 & 100 & 1.70264 & 0.000931665 & 0.000416653 \\
40 & 110 & 1.91394 & 0.00271256  & 0.0012131 \\
\bottomrule
\end{tabular}
    \caption{Medie tempi a 45'}
    \label{tab:45_primi}
\end{table}

\begin{table}[h]
\centering
\begin{tabular}{cc|ccc}
\toprule
Inizio&Fine&$\overline{t}$&$\sigma$&$\sigma_{\overline{t}}$\\
$[\si{cm}]$&[cm]&[sec]&[sec]&[sec]\\
\midrule
40 & 50  & 0.37292 & 0.000216795 & 9.69536e-05 \\
40 & 60  & 0.6915  & 0.000648074 & 0.000289828 \\
40 & 70  & 0.97826 & 0.000397492 & 0.000177764 \\
40 & 80  & 1.23896 & 0.00131453  & 0.000587878 \\
40 & 90  & 1.4787  & 0.001       & 0.000447214 \\
40 & 100 & 1.70254 & 0.000364692 & 0.000163095 \\
40 & 110 & 1.91188 & 0.00130652  & 0.000584294 \\
\bottomrule
\end{tabular}
    \caption{Medie tempi a 45' con massa}
    \label{tab:d45_primi}
\end{table}


Si è proceduto con il calcolo della velocità media per ogni intervallo di spazio di 10cm, partendo dall'intervallo [40cm-50cm] e terminando con l'intervallo [100cm-110cm] utilizzando la seguente formula:
\begin{equation*}
    \overline{v_{(x_i;x_{i+1})}}=\frac{x_{i+1}-x_i}{t(40;x_{i+1})-t(40;x_i)}
\end{equation*}
assumendo come valori di $x_i$ tutti i possibili estremi degli intervalli considerati, ovvero 40cm,50cm,60cm,...,110cm.

Utilizzando il teorema delle varianze si è poi calcolato l'errore casuale della velocità media per ogni intervallo, assumendo come errore sul tempo la deviazione standard della media di ciascun intervallo precedentemente calcolato, e come errore sullo spazio la deviazione standard relativa alla distribuzione triangolare, utilizzando le seguenti :

\begin{equation*}
    \sigma_{v}= \sqrt{
   \left( {\frac{\partial v}{\partial x_{i}}} \Big|_{\ast}\right)^{2} \cdot \sigma_{x_{i}}^2  +   
   \left( {\frac{\partial v}{\partial x_{i+1}}}\Big|_{\ast}\right )^{2} \cdot \sigma_{x_{i+1}}^2   +   
   \left( {\frac{\partial v}{\partial t_{\left(40; x_{i+1}\right)}}}\Big|_{\ast}\right)^{2}  \cdot \sigma_{t_{\left(40; x_{i+1}\right)}}^2 +   
       \left( {\frac{\partial v}{\partial t_{\left(40; x_i\right)}}}\Big|_{\ast}\right)^{2} \cdot  \sigma_{t_{\left(40; x_i\right)}}^2
    }
\end{equation*}

\begin{equation*}
\begin{multilined}
   \newline \sigma_{x}=\frac{\mathopen|2\Delta x\mathclose|}{\sqrt{24}} 

   \newline \text{Dove $\Delta x$ è la più piccola tacca di misura leggibile dall'operatore sulla stecca metrica}
\end{multilined}
\end{equation*}

Considerando le coppie di dati tempi intermedi e velocità medie corrispondenti precedentemente calcolati, queste ultime sono state utilizzate come coordinate di punti su un piano cartesiano, e si è ricercata una loro interpolazione lineare sfruttando il metodo del minimo ${\chi}^2$, ottenendo il coefficiente angolare, corrispondente alla componente dell'accelerazione di gravità g sulla parallela  alla guidovia, e l'intercetta della retta cercata, corrispondente alla velocità della slitta nella posizione iniziale, 40cm.\\
Per stimare $g_{0}$ è stato sufficiente dividere il coefficiente angolare della retta per il seno dell'angolo di inclinazione della guida.
Per il calcolo di $ \sigma_{g_{0}}$ si è scelto di calcolare l' errore relativo all'angolo utilizzando la distribuzione triangolare, l'errore relativo a $\sigma_b$ con il metodo del minimo ${\chi}^2$ per poi propagarli tramite il teorema delle varianze. Si è proceduto infine con il calcolo della compatibilità delle 4 stime dell'accelerazione di gravità $g_{0}$ relative ai 4 differenti campioni con $g=(9.801\pm 0.001)\si{m/s^2}$ stimata a Padova.\\
Tutto il procedimento è stato ripetuto variando l'inclinazione della guida portandola dapprima a 30', poi a 45' ed infine mantenendo l'inclinazione della guida a 45' si è aumentata la massa della slitta con un disco di ottone.\\


  


\begin{figure}[h]
    \caption{Interpolazione lineare a 15'}
    \label{fig:g_0_15}
    \centering
           \includegraphics[width=15cm]{15primi.png}
\end{figure}

\begin{figure}[h]
    \caption{Interpolazione lineare a 30'}
    \label{fig:g_0_30}
    \centering
           \includegraphics[width=15cm]{30primi.png}
\end{figure}

\begin{figure}[h]
    \caption{Interpolazione lineare a 45'}
    \label{fig:g_0_45}
    \centering
           \includegraphics[width=15cm]{45primi.png}
\end{figure}

\begin{figure}[h]
    \caption{Interpolazione lineare a 45' con massa}
    \label{fig:g_0_p45}
    \centering
           \includegraphics[width=15cm]{45primi_p.png}
\end{figure}



\begin{table}[]
\caption{Stime di accelerazione $b\pm \sigma_{b}$, di gravità $g_{0} \pm \sigma_{g_{0}}$ e relativa compatibilità $\lambda$ con g stimata a Padova}
\label{tab:stima_b_g}
\begin{tabular}{r|c|c|c|c}
\hline
\multicolumn{1}{l|}{}  & \textbf{15'}              & \textbf{30'}              & \textbf{45'}              & \textbf{45' con Massa}    \\ \hline
\textbf{Accelerazione slitta $[\si{m/s^2}]$} & $0.04\pm0.01$    & $0.08\pm0.03$   & $0.12\pm0.05$    & $0.12\pm0.05$    \\
\textbf{Gravità $[\si{m/s^2}]$}       & $9.45\pm0.06$ & $9.58\pm0.07$ & $9.51\pm0.07$ & $9.78\pm0.07$ \\ 
\textbf{Compatibilità $\lambda$}              &5.1                &3.08           &3.9            &0.22\\ \hline
\end{tabular}
\end{table}



\subsection{Seconda esperienza}

\section{Discussione dei risultati}
\section{Margini di miglioramento}
\section{Conclusione}
\newpage
\section{Appendice}
\subsection{Formulario}
\subsection{Dati grezzi}
\subsection{Codice sorgente}




\end{document}

Premettendo che la compatibilità $\lambda$ sia:
\begin{equation*}\label{eq:cases}
    \begin{cases}
    0<\lambda\leq1, & \text{Ottima}\\
    1<\lambda\leq2, & \text{Discreta}\\
    2<\lambda\leq3, & \text{Pessima}\\
    3<\lambda, & \text{Non compatibile}\\
    \end{cases}
\end{equation*}
